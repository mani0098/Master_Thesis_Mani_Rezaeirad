\cleardoublepage\phantomsection\addcontentsline{toc}{chapter}{Abstract}
\chapter*{Abstract}

\acrfull{ML} has become a part of our lives. In medicine, particularly, it is utilized more frequently to aid medical treatments. There are two separate areas within medicine where ML methods are studied today: medical imaging and prediction (clinical decision support) that are based on demographic/discrete variables (e.g., age, diagnosis, clinical findings, blood tests) and continuous vitals (e.g., electrocardiography, blood pressure, and electroencephalography). The latter being the focus of this thesis. Regarding prediction (clinical decision support), there are challenges related to the human body's complex behavior. Specifically, to apply the prediction of mortality in medical treatments, i.e., connecting the consequences of changes in vital signals to mortality predictions.\

{\hskip 1em} Many groups are working with data from PhysioNet, and especially with the \acrshort{eICU} database. This is a database of about 200,000 \acrfull{ICU} admitted patients; however, extracting specific information from the database is not-trivial since there is a disorder in sampling frequency, what variables are sampled, and so forth. Therefore, data science is required to preprocess the data. The data are labeled with several outcome variables, among them death. \

{\hskip 1em} This thesis focuses on presenting a statistical model that calculates the possibility of mortality mainly based on the measurements of glucose and lactate at distinct time intervals or tracing individuals from day to day. Utilizing the model to predict changes in some selected features (weight, glucose, and lactate levels) as changes in those can influence mortality probability. \

{\hskip 1em} Several ML methods have been used, and the performance of each has been assessed and compared. Ultimately, it can be seen that the \acrlong{XGB} algorithm outperforms the other six algorithms used for mortality prediction in three metrics, accuracy, area under the curve, and Cohen's Kappa. This type of modeling provides feedback to the doctor saying whether this unique patient is improving or deteriorating compared to the overall ICU stay; hence, connecting the predictions to the treatment in a more efficient way.